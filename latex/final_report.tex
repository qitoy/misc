\documentclass[dvipdfmx]{beamer}
\usepackage{mathtools}
\mathtoolsset{showonlyrefs=true}
\usetheme{Boadilla}
\usefonttheme{professionalfonts}
\mathversion{bold}
\renewcommand{\kanjifamilydefault}{\gtdefault}
\setbeamertemplate{navigation symbols}{}

\title{不思議な恒等式の紹介}
\author{qitoy}
\date{}

\begin{document}
\maketitle

\begin{frame}
	突然だが、以下の数式を計算するとどうなるだろうか。
	\begin{align}
		  & n^2 - 3(n+1)^2 + 3(n+2)^2 - (n+3)^2                      \\
		\uncover<2-> {
		= & n^2 - 3(n^2 + 2n + 1) + 3(n^2 + 4n + 4) - (n^2 + 6n + 9) \\
		= & 0
		}
	\end{align}
	\uncover<2-> {
		展開して計算すると、恒等的に$0$となる。
	}
	\uncover<3-> {

		また、以下の恒等式も成立する。
		\begin{gather}
			n^3-4(n+1)^3+6(n+2)^3-4(n+3)^3+(n+4)^3 = 0 \\
			\begin{multlined}
				n^4-5(n+1)^4+10(n+2)^4 \\
				-10(n+3)^4+5(n+4)^4-(n+5)^4 = 0
			\end{multlined}
		\end{gather}
	}
\end{frame}

\begin{frame}
  これらの結果の一般化として、以下の恒等式が成立することが証明できる。
	\begin{block}{恒等式}
		\[
			\sum_{j=0}^{k+1} (-1)^j \binom{k+1}{j} (n+j)^k = 0 \quad (k \in \mathbb{Z}_{\ge 0})
		\]
	\end{block}
	ここでは$n^k$の差分を利用する証明を紹介しよう。
\end{frame}

\begin{frame}{証明}
	$a_n \coloneq n^k$とする。また、数列の差分をとる(つまり、$a_n$に対して$a_{n+1}-a_n$を計算する)演算子を$D$とおく。多項式に差分を適応すると微分のように次数が$1$減ることから、$D^{k+1}(a_n) = D^{k+1}(n^k) = 0$である。%
	\uncover<2-> {%
		一方で$D^{k+1}(a_n)$について考えると、
		\begin{align}
			D(a_n)   & = a_{n+1} - a_n                                \\
			D^2(a_n) & = D(a_{n+1} - a_n) = a_{n+2} - 2 a_{n+1} + a_n \\
			D^3(a_n) & = a_{n+3} - 3 a_{n+2} + 3 a_{n+1} - a_n        \\
			         & \vdots
		\end{align}
		となり帰納的に$D^{k+1}(a_n) = \sum_{j=0}^{k+1} (-1)^j \binom{k+1}{j} a_{n+k+1-j}$が得られる。これに$a_n = n^k$を代入し整理すると求める式となる。\qed
	}
\end{frame}

\begin{frame}
	また、この恒等式から以下の等式が導かれる。
	\begin{block}{}
		\[
			\sum_{j=0}^{k+1} (-1)^j j^l \binom{k+1}{j} = 0 \quad (l = 0, \ldots, k)
		\]
	\end{block}
	\pause
	\begin{proof}[証明]
		\begin{align}
			  & \sum_{j=0}^{k+1} (-1)^j \binom{k+1}{j} (n+j)^k \\
			= & \sum_{j=0}^{k+1} (-1)^j \binom{k+1}{j} \left( \sum_{l=0}^k \binom{k}{l} n^{k-l} j^l \right)     \\
			= & \sum_{l=0}^k \binom{k}{l} \left( \sum_{j=0}^{k+1} (-1)^j j^l \binom{k+1}{j} \right) n^{k-l} = 0
		\end{align}
		であり、両辺の$n^{k-l}$の係数を比較して等式を得る。
	\end{proof}
\end{frame}

\begin{frame}{余談}
	\begin{block}{恒等式(再掲)}
		\[
			\sum_{j=0}^{k+1} (-1)^j \binom{k+1}{j} (n+j)^k = 0 \quad (k \in \mathbb{Z}_{\ge 0})
		\]
	\end{block}
	今回この恒等式を発見したきっかけは$n^2$がみたす$m$項間漸化式を求めたことだった。証明でしたように差分を複数回($k=2$なので$3$回)とることで$a_{n+3} - 3 a_{n+2} + 3 a_{n+1} - a_n = 0$と$m$項間漸化式($m=4$)が得られる。漸化式に一般項を代入すると等式が成立するのは当然ではあるものの、代入してできたこの恒等式の非自明さに惹かれ、紹介することにした。
\end{frame}

\end{document}
